%%%%%%%%%%%%%%%%%%%%%%%%%%%%%%%%%%%%%%%%%
% Beamer Presentation
% LaTeX Template
% Version 1.0 (10/11/12)
%
% This template has been downloaded from:
% http://www.LaTeXTemplates.com
%
% License:
% CC BY-NC-SA 3.0 (http://creativecommons.org/licenses/by-nc-sa/3.0/)
%
%%%%%%%%%%%%%%%%%%%%%%%%%%%%%%%%%%%%%%%%%

%-------------------------------------------------------------------------------
%	PACKAGES AND THEMES
%-------------------------------------------------------------------------------

\documentclass{beamer}

\mode<presentation> {

% The Beamer class comes with a number of default slide themes
% which change the colors and layouts of slides. Below this is a list
% of all the themes, uncomment each in turn to see what they look like.

%\usetheme{default}
%\usetheme{AnnArbor}
%\usetheme{Antibes}
%\usetheme{Bergen}
%\usetheme{Berkeley}
%\usetheme{Berlin}
%\usetheme{Boadilla}
%\usetheme{CambridgeUS}
%\usetheme{Copenhagen}
%\usetheme{Darmstadt}
%\usetheme{Dresden}
%\usetheme{Frankfurt}
%\usetheme{Goettingen}
%\usetheme{Hannover}
%\usetheme{Ilmenau}
%\usetheme{JuanLesPins}
%\usetheme{Luebeck}
%\usetheme{Madrid}
%\usetheme{Malmoe}
%\usetheme{Marburg}
%\usetheme{Montpellier}
%\usetheme{PaloAlto}
%\usetheme{Pittsburgh}
%\usetheme{Rochester}
%\usetheme{Singapore}
%\usetheme{Szeged}
\usetheme{Warsaw}

% As well as themes, the Beamer class has a number of color themes
% for any slide theme. Uncomment each of these in turn to see how it
% changes the colors of your current slide theme.

%\usecolortheme{albatross}
%\usecolortheme{beaver}
%\usecolortheme{beetle}
%\usecolortheme{crane}
%\usecolortheme{dolphin}
%\usecolortheme{dove}
%\usecolortheme{fly}
%\usecolortheme{lily}
\usecolortheme{orchid}
%\usecolortheme{rose}
%\usecolortheme{seagull}
%\usecolortheme{seahorse}
%\usecolortheme{whale}
%\usecolortheme{wolverine}

%\setbeamertemplate{footline} % To remove the footer line in all slides
%\setbeamertemplate{footline}[page number] % To replace the footer line

%\setbeamertemplate{navigation symbols}{} % To remove the navigation symbols from the bottom of all slides uncomment this line
}

\usepackage{graphicx} % Allows including images
\usepackage{booktabs} % Allows the use of \toprule, \midrule and \bottomrule in tables

%----------------------------------------------------------------------------------------
%	TITLE PAGE
%----------------------------------------------------------------------------------------

\title[Analyzing CA Education Trends]{Analyzing California Education Trends}

\author{Neal Marquez} % Your name
\institute[UW] % Your institution as it will appear on the bottom of every slide, may be shorthand to save space
{
University of Washington \\ % Your institution for the title page
\medskip
\textit{nmarquez@uw.com} % Your email address
}
\date{December 5, 2017} % Date, can be changed to a custom date

\begin{document}

\begin{frame}
\titlepage % Print the title page as the first slide
\end{frame}

\begin{frame}
\frametitle{Overview} % Table of contents slide, comment this block out to del
\tableofcontents % Throughout your presentation,
% if you choose to use \section{} and \subsection{} commands, these will
% automatically be printed on this slide as an overview of your presentation
\end{frame}

%-------------------------------------------------------------------------------
%	PRESENTATION SLIDES
%-------------------------------------------------------------------------------

%------------------------------------------------
\section{Background} % Sections can be created in order to organize your
%------------------------------------------------

% \subsection{Subsection Example} % A subsection can be created just before a set
% of slides with a common theme to further break down your presentation into chunks

\begin{frame}
\frametitle{Education Trends for Minority Students}
\begin{itemize}
\item Past 15 years have seen increase in minority representation in higher education
\begin{itemize}
\item True for both rates and ablsolute numbers
\end{itemize}
\item Despite this there continues to be unequal representation by various factors
\begin{itemize}
\item Time till graduation
\item Graduation rates
\item Differnces of attainment by economic status
\end{itemize}
\end{itemize}
\end{frame}

%------------------------------------------------

\begin{frame}
\frametitle{Historical Trends}
\begin{figure}
\includegraphics[width=0.8\linewidth]{../data/plots/pew.png}
\end{figure}
\end{frame}

%------------------------------------------------

\begin{frame}
\frametitle{Motivation}
\begin{itemize}
  \item How does this pattern play out for more prestegious universities?
  \begin{itemize}
    \item Some evidence with Ivy League schools
  \end{itemize}
  \item Does the composition of the student demogrpahy play a role?
  \begin{itemize}
    \item Do students benfit when demographies reflect their own or others?
  \end{itemize}
  \item How have these effects changed over time?
\end{itemize}
\end{frame}

%------------------------------------------------

\begin{frame}
\frametitle{Mechanisms of Operation}
\begin{block}{Economic Investment}
Different levels of investment in education may reflect different levels of
acceptance into prestigious academic universities by way of preparing and
motivating students.
\end{block}

\begin{block}{Role Theory}
Individuals take place in certain roles that are legitamized throush social
interaction. Students who are part of groups that have historically been
associated with high achievement.
\end{block}

\begin{block}{Evolving Home Environment}
Individual's circumstances at home are differential by different race and class
identifiers. The environment may be changing in a beneficial, ore detrimental,
manner allowing more time for academic pursuit.
\end{block}
\end{frame}

%------------------------------------------------

\begin{frame}
\frametitle{Hypothesis Tests}
How does the proportion of a particular racial group alter the chances of a
hispanic student being accepted to a prestigious university?  \\
~ \\
How has this effect changed over time and how much does it differ across
schools for hispanic students?
\end{frame}

%------------------------------------------------
\section{Data}
%------------------------------------------------

\begin{frame}
\frametitle{Data}
\begin{itemize}
  \item High school demographic data pulled from DOE CA
  \begin{itemize}
    \item Population counts of graduating class by racial groups
    \item Data span range 1994-2015
  \end{itemize}
  \item University of California application data
  \begin{itemize}
    \item Counts of applied, admitted, and accepted populations by racial groups
    \item Data span range 1994-2015
  \end{itemize}
\end{itemize}
\end{frame}

%------------------------------------------------

\begin{frame}
\frametitle{Data Merging}
\begin{columns}[c] % The "c" option specifies centered vertical alignment

\column{.5\textwidth} % Left column and width
\begin{itemize}
\item Data for high school changes racial groups
\item Naming conventions are not ubiquitous
\item No guarantee for consistent identification
\end{itemize}

\column{.5\textwidth} % Right column and width
\begin{figure}
\includegraphics[width=0.8\linewidth]{../data/plots/datascreengrab.png}
\end{figure}
\end{columns}
\end{frame}

%------------------------------------------------

\begin{frame}
\frametitle{Variables in Model}

\begin{itemize}
  \item Response
  \begin{itemize}
    \item Counts of Hispanic students admitted into UC institutions for school(i) year (j)
  \end{itemize}
  \item Variables
    \begin{itemize}
      \item Time scaled ($(2015 - x) / 10$)
      \item Proportion of White students in school at year of graduation.
    \end{itemize}
  \item Offset
  \begin{itemize}
    \item Student Hispanic population count
    \item Applying Hispanic student count
  \end{itemize}
\end{itemize}

\end{frame}

%------------------------------------------------

\begin{frame}
\frametitle{Sampled Trends}
\begin{figure}
\includegraphics[width=0.8\linewidth]{../data/plots/sample_plot.png}
\end{figure}
\end{frame}

%------------------------------------------------
\section{Model Building}
%------------------------------------------------

\begin{frame}
\frametitle{Functional Form}
\begin{block}{Model Family Form}
  \begin{itemize}
    \item $\hat{\theta}_{ij} = exp(\boldsymbol{\beta}\mathbf{X} + \mathbf{ZY}) \mathbf{E}$
    \begin{enumerate}
      \item Poisson($\theta$)
      \begin{itemize}
        \item Mean $= \theta$
        \item Var $= \theta$
      \end{itemize}
      \item NB2($\theta$, $\psi$)
      \begin{itemize}
        \item Mean $= \theta$
        \item Var $= \theta + \frac{\theta^2}{\psi}$
      \end{itemize}
    \end{enumerate}
  \end{itemize}
\end{block}
\end{frame}

%------------------------------------------------

%\begin{frame}[fragile] % Need to use the fragile option when verbatim is used in the slide
\begin{frame}
\frametitle{Model Testing: $\boldsymbol{\beta}\mathbf{X} + \mathbf{ZY}$}
\begin{enumerate}
  \item $\beta_0 + \text{time}_{ij} \beta_1  + \text{pwhite}_{ij} \beta_2$
  \item $\beta_0 + \text{time}_{ij} \beta_1  + \text{pwhite}_{ij} \beta_2 + \text{pwhite}_{ij}\text{time}_{ij} \beta_3$
  \item $\beta_0 + \text{time}_{ij} \beta_1  + \text{pwhite}_{ij} \beta_2 + \text{pwhite}_{ij}\text{time}_{ij} \beta_3 + \zeta_{0j}$
  \item $\beta_0 + \text{time}_{ij} \beta_1  + \text{pwhite}_{ij} \beta_2 + \text{pwhite}_{ij}\text{time}_{ij} \beta_3 + \zeta_{0i} + \text{time}_{ij} \zeta_{1i}$
\end{enumerate}
\end{frame}

%------------------------------------------------

\begin{frame}
  \frametitle{Variance Structure}
  Random Effects Structure \\
  ~ \\
\begin{flalign*}
\begin{bmatrix}
   \zeta_{0i} \\
   \zeta_{1i} \\
\end{bmatrix}
\
\sim \mathcal{N} \Bigg(
\
\begin{bmatrix}
   0 \\
   0 \\
\end{bmatrix}
\
,
\begin{bmatrix}
   \sigma_0^2 & \sigma_{01} \\
   \sigma_{01} & \sigma_1^2 \\
\end{bmatrix}
\
\Bigg)
\end{flalign*}
\end{frame}
%------------------------------------------------

\begin{frame}
\frametitle{Model Comparison Tests}

\begin{tabular}{l|r|r|r|r|r}
\hline
  & Df & AIC & BIC & Chisq & Pr\\
\hline
Model1 & 4 & 66206.67 & 66235.50 & NA & NA\\
\hline
Model2 & 5 & 65998.90 & 66034.94 & 209.7641 & 0\\
\hline
Model3 & 6 & 58867.16 & 58910.41 & 7133.7411 & 0\\
\hline
Model4 & 8 & 58329.97 & 58387.62 & 541.1961 & 0\\
\hline
\end{tabular}

\end{frame}

%------------------------------------------------
\section{Model Diagnostics}
%------------------------------------------------

\begin{frame}
\frametitle{Residuals}
\begin{columns}[c] % The "c" option specifies centered vertical alignment

\column{.5\textwidth} % Right column and width
\begin{figure}
\includegraphics[width=0.8\linewidth]{../data/plots/level1_residuals.png}
\end{figure}

\column{.5\textwidth} % Right column and width
\begin{figure}
\includegraphics[width=0.8\linewidth]{../data/plots/temporalresiduals.png}
\end{figure}

\end{columns}
\end{frame}

%------------------------------------------------

\begin{frame}
\frametitle{Random Effect Distribution}
\begin{figure}
\includegraphics[width=0.8\linewidth]{../data/plots/model1RE.png}
\end{figure}
\end{frame}

%------------------------------------------------
\section{Results}
%------------------------------------------------

\begin{frame}
\frametitle{Population-Admitted}
\begin{columns}[c] % The "c" option specifies centered vertical alignment

\column{.58\textwidth} % Right column and width
\begin{figure}
\includegraphics[width=1\linewidth]{../data/plots/avgpredcadm.png}
\end{figure}

\column{.58\textwidth} % Right column and width
\begin{figure}
\includegraphics[width=1\linewidth]{../data/plots/simspredcadm.png}
\end{figure}

\end{columns}
\end{frame}

%------------------------------------------------

\begin{frame}
\frametitle{Population-Applied}
\begin{columns}[c] % The "c" option specifies centered vertical alignment

\column{.58\textwidth} % Right column and width
\begin{figure}
\includegraphics[width=1\linewidth]{../data/plots/avgpredcapp.png}
\end{figure}

\column{.58\textwidth} % Right column and width
\begin{figure}
\includegraphics[width=1\linewidth]{../data/plots/simspredcapp.png}
\end{figure}

\end{columns}
\end{frame}

%------------------------------------------------

\begin{frame}
\frametitle{Applied-Admitted}
\begin{columns}[c] % The "c" option specifies centered vertical alignment

\column{.58\textwidth} % Right column and width
\begin{figure}
\includegraphics[width=1\linewidth]{../data/plots/avgpredappadm.png}
\end{figure}

\column{.58\textwidth} % Right column and width
\begin{figure}
\includegraphics[width=1\linewidth]{../data/plots/simspredappadm.png}
\end{figure}

\end{columns}
\end{frame}

%------------------------------------------------

\begin{frame}
\frametitle{Hsipanic Population Covariate}
\begin{columns}[c] % The "c" option specifies centered vertical alignment

\column{.58\textwidth} % Right column and width
\begin{figure}
\includegraphics[width=1\linewidth]{../data/plots/havgpredcapp.png}
\end{figure}

\column{.58\textwidth} % Right column and width
\begin{figure}
\includegraphics[width=1\linewidth]{../data/plots/havgpredapadm.png}
\end{figure}

\end{columns}
\end{frame}

%------------------------------------------------
\begin{frame}
\Huge{\centerline{The End}}
\end{frame}

%----------------------------------------------------------------------------------------

\end{document}
